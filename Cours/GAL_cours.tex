\documentclass[a4paper, 12pt, usenames, svgnames, chapterprefix=true]{scrreprt}

\usepackage[utf8]{inputenc}
\usepackage[T1]{fontenc}
\usepackage{graphicx, wrapfig}
\usepackage{lmodern}
\usepackage{color, colortbl}
\usepackage{xcolor}
\usepackage{lipsum}
\usepackage{amsmath, amssymb, mathrsfs, amsthm, thmtools, MnSymbol}
\usepackage[framemethod=tikz]{mdframed}
\usepackage{pgf, pgfplots, tikz, pst-solides3d}
\usetikzlibrary{cd} %To draw commutative diagrams
\usetikzlibrary{calc}
\usetikzlibrary{arrows}
\usepackage[chapter]{algorithm}
\usepackage{algorithmicx, algpseudocode}
\usepackage{listings}
\usepackage{multicol, multirow}
% the following patch corrects a bug for the closing parenthesis  
\usepackage{etoolbox}
\makeatletter
\patchcmd{\lsthk@SelectCharTable}{%
  \lst@ifbreaklines\lst@Def{`)}{\lst@breakProcessOther)}\fi}{}{}{}
\makeatother
\usepackage{hyperref}
\usepackage{todonotes}
\usepackage{makeidx}
\usepackage[inline]{enumitem}
\usepackage[francais]{babel}
\usepackage{caption, tabu}
\usepackage[headsepline=1pt,plainheadsepline,
  footsepline=1pt,plainfootsepline]{scrlayer-scrpage}  % header and footer for KOMA-Script

% ---- User defined commands
\newcommand{\N}{\mathbb{N}}
\newcommand{\Q}{\mathbb{Q}} 
\newcommand{\R}{\mathbb{R}}
\newcommand{\Z}{\mathbb{Z}}
\newcommand{\F}{\mathbb{F}}
\newcommand{\Fa}{\F(A)} 
\newcommand{\C}{\mathbb{C}}
\newcommand{\K}{\mathbb{K}}
\renewcommand{\epsilon}{\varepsilon}
\renewcommand{\phi}{\varphi}
\renewcommand{\emph}{\textbf}
\newcommand{\im}{\mathrm{Im}}

% ---- Synchronization with Skim
\synctex=1


%%%%%%%%	Définitions des environnements de théorèmes	%%%%%%%%
%----- ENVIRONNEMENT POUR LES DÉFINITIONS ----%
\declaretheoremstyle[
  spaceabove=0pt, spacebelow=0pt, headfont=\normalfont\bfseries\scshape,
    notefont=\mdseries, notebraces={(}{)}, headpunct={. }, headindent={},
    postheadspace={ }, postheadspace=4pt, bodyfont=\normalfont, %qed=$$,
    mdframed={
      leftmargin=-5,
      rightmargin=-5,
      middlelinewidth=1pt,
      roundcorner=5pt,
      middlelinecolor=DarkSlateGrey,
      innerlinecolor=DarkSlateGrey,
      outerlinecolor=DarkSlateGrey,
      % apptotikzsetting={\tikzset{mdfbackground/.append style ={
      %       shade, left color=DarkSlateGrey!20, right color = DarkSlateGrey!20}}}
   }
]{defstyle}

\declaretheorem[style=defstyle, numberwithin=chapter, title=Définition]{defi}
%________________________________________________________


%----- ENVIRONNEMENT POUR LES THEOREMES ----%
\declaretheoremstyle[
  spaceabove=0pt, spacebelow=0pt, headfont=\normalfont\bfseries\scshape,
    notefont=\mdseries, notebraces={(}{)}, headpunct={. }, headindent={},
    postheadspace={ }, postheadspace=4pt, bodyfont=\normalfont\itshape, %qed=$$,
    mdframed={
      leftmargin=-5,
      rightmargin=-5,
      middlelinewidth=1pt,
      roundcorner=5pt,
      middlelinecolor=DarkOrange,
      innerlinecolor=DarkOrange,
      outerlinecolor=DarkOrange,
      % apptotikzsetting={\tikzset{mdfbackground/.append style ={
      %       shade, left color=DarkSlateGrey!20, right color = DarkSlateGrey!20}}}
   }
]{thmstyle}
\declaretheorem[style=thmstyle, sibling=defi, title=Théorème]{theo}
\declaretheorem[style=thmstyle, sibling=defi, title=Corollaire]{cor}
\declaretheorem[style=thmstyle, sibling=defi, title=Proposition]{prop}
\declaretheorem[style=thmstyle, sibling=defi, title=Propriétés]{propri}
\declaretheorem[style=thmstyle, sibling=defi, title=Observation]{obs}
\declaretheorem[style=thmstyle, sibling=defi, title=Observations]{obss}
\declaretheorem[style=thmstyle, sibling=defi, title=Lemme]{lem}
\declaretheorem[style=thmstyle, sibling=defi, title=Conséquence]{conseq}
%_________________________________________________________

%----- ENVIRONNEMENT POUR LES PREUVES ----%
\declaretheoremstyle[
  spaceabove=0pt, spacebelow=0pt, headfont=\normalfont\bfseries\scshape,
    notefont=\mdseries, notebraces={(}{)}, headpunct={. }, headindent={},
    postheadspace={ }, postheadspace=4pt, bodyfont=\normalfont, 
    mdframed={
      leftmargin=15,
      rightmargin=15,
      hidealllines=true,
      font=\small
   }
]{preuvestyle}

\declaretheorem[style=preuvestyle, numbered = no, title=Preuve, qed=\textcolor{DarkSlateGrey!80}{\qedsymbol}]{preuve}
\declaretheorem[style=preuvestyle, title=Exercice, numberwithin=chapter, qed=\textcolor{DarkSlateGrey!80}{$\spadesuit$}]{exercice}
\declaretheorem[style=preuvestyle, sibling=defi, title=Remarque, qed =
\textcolor{DarkSlateGrey!80}{$\clubsuit$}]{rem}
\declaretheorem[style=preuvestyle, sibling=defi, title=Remarques, qed = \textcolor{DarkSlateGrey!80}{$\clubsuit$}]{rems}
%________________________________________________________
%----- ENVIRONNEMENT POUR LES EXEMPLES ----%
\declaretheoremstyle[
  spaceabove=0pt, spacebelow=0pt, headfont=\normalfont\bfseries\scshape,
    notefont=\mdseries, notebraces={(}{)}, headpunct={. }, headindent={},
    postheadspace={ }, postheadspace=4pt, bodyfont=\normalfont, qed=\textcolor{DarkOliveGreen!80}{$\bigstar$},
    mdframed={
      leftmargin=15,
      rightmargin=15,
      font=\small,
      outerlinewidth=1pt,
      innerlinewidth=1pt,
      middlelinewidth=1pt,
      hidealllines=true,
      topline=true,
      bottomline=true,
      innerlinecolor=DarkOliveGreen!80,
      outerlinecolor=DarkOliveGreen!80,
      middlelinecolor=DarkOliveGreen!80
      % , leftline=true,
      %innerlinecolor=DarkSlateGrey!80,
      %outerlinecolor=DarkSlateGrey!80,
      %middlelinecolor=White,
   }
]{exstyle}

\declaretheorem[style=exstyle, numberlike=defi, title=Exemple]{ex}
\declaretheorem[style=exstyle, numberlike=defi, title=Exemples]{exs}
%________________________________________________________

% ---- Headers and footers
\clearpairofpagestyles                 % deletes header/footer
\pagestyle{scrheadings}           % use following definitions for header/footer
% definitions/configuration for the header
\lohead[Université de \textsc{Neuchâtel}]{Université de \textsc{Neuchâtel}}
\lehead[Université de \textsc{Neuchâtel}]{Université de \textsc{Neuchâtel}}        % equal page, right position (inner) 
\rohead[\leftmark]{\leftmark}        % odd   page, left  position (inner) 
\rehead[\rightmark]{\rightmark} % equal page, left (outer) position
% definitions/configuration for the footer
\lefoot[Théorie de Galois]{Théorie de Galois}
\lofoot[Théorie de Galois]{Théorie de Galois}
\refoot[page \pagemark]{page \pagemark}
\rofoot[page \pagemark]{page \pagemark}
\renewcommand{\chaptermark}[1]{%
  \markboth{#1}{}}


\addtokomafont{disposition}{\normalfont\bfseries}

\title{\normalfont{\bfseries{Théorie de Galois \\ {\Large printemps 2017} \\{\large Université de Neuchâtel}}}}
\author{Enseigné par Ana \textsc{Khukhro}\\Notes prises par Laurent \textsc{Hayez}}
\date{Date de création: 23 février 2017\\ Dernière modification: \today}

\makeindex

\begin{document}


\renewcommand{\labelitemi}{\textbullet}

\tikzset{math3d/.style=
{x= {(-0.353cm,-0.353cm)}, y={(1cm,0cm)}, z={(0cm,1cm)}}}


\maketitle


%Table of contents
\tableofcontents

% Chapter 1: Introduction
% ********************************************************************************************************** %
% *                                            THÉORIE DE GALOIS                                           * %
% *                                            -----------------                                           * %
% *                                                                                                        * %
% * Chapitre 0: Introduction                                                                               * %
% *                                                                                                        * %
% * Auteur: Laurent Hayez                                                                                  * %
% * Cours enseigné au semestre de printemps 2017 à l'université de Neuchâtel par Ana Khukhro               * %
% ********************************************************************************************************** %

\chapter{Introduction et histoire}
Babylone vers 1600 av. J.-C., solution de l'équation du second degré. 
\[ ax^2 + bx + c = 0 \iff x = \frac{-b \pm \sqrt{b^2 - 4ac}}{2a}. \]
En $\sim\!$1540 ap. J.-C., Caradano et Ferrari donnent une solution pour les polynômes de degré jusqu'à $4$. 
La question restait ouverte pour les polynômes de degrés plus grands ou égal à $5$. \og Résolubles par
radicaux\fg{}? Évariste \textsc{Galois} (1811-1832) donne la solution. 

Si $p(x)$ est un polynôme et $E$ une extension d'un corps $K$ $\ext{E}{K}$, son idée est de construire un
groupe à partir de $E$.

Il y a également des constructions à la règle et au compas (Grèce: Euclide, etc. en 300/400
av. J.-C. environ), par exemple donner l'ensemble des points équidistants à un point $A$ et un point $B$. Il y
a des questions que les grecs n'ont pas réussi à résoudre, par exemple
\begin{itemize}
\item la trisection de l'angle (partager un angle en 3);
\item la quadrature du cercle;
\item la duplication du cube (cube donné, trouver un cube plus grand ayant le double du volume).
\end{itemize}





  




%%% Local Variables:
%%% mode: latex
%%% TeX-master: "../GAL_cours.tex" 
%%% End:

% Chapter 2: Présentations des groupes
% ********************************************************************************************************** %
% *                                            THÉORIE DE GALOIS                                           * %
% *                                            -----------------                                           * %
% *                                                                                                        * %
% * Chapitre 2:                                                                                * %
% *                                                                                                        * %
% * Auteur: Laurent Hayez                                                                                  * %
% * Cours enseigné au semestre de printemps 2017 à l'université de Neuchâtel par Ana Khukhro               * %
% ********************************************************************************************************** %

\chapter{}






%%% Local Variables:
%%% mode: latex
%%% TeX-master: "../GAL_cours.tex" 
%%% End:




\printindex
	
\end{document}



%%% Local Variables:
%%% mode: latex
%%% TeX-master: t 
%%% End: