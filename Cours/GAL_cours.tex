\documentclass[a4paper, 12pt, usenames, svgnames, chapterprefix=true]{scrreprt}

\usepackage[utf8]{inputenc}
\usepackage[T1]{fontenc}
\usepackage{graphicx, wrapfig}
\usepackage{lmodern}
\usepackage{color, colortbl}
\usepackage{xcolor}
\usepackage{lipsum}
\usepackage{amsmath, amssymb, mathrsfs, amsthm, thmtools, MnSymbol}
\usepackage[framemethod=tikz]{mdframed}
\usepackage{pgf, pgfplots, tikz, pst-solides3d}
\usetikzlibrary{cd} %To draw commutative diagrams
\usetikzlibrary{calc}
\usetikzlibrary{arrows}
\usepackage[chapter]{algorithm}
\usepackage{algorithmicx, algpseudocode}
\usepackage{listings}
\usepackage{multicol, multirow}
% the following patch corrects a bug for the closing parenthesis  
\usepackage{etoolbox}
\makeatletter
\patchcmd{\lsthk@SelectCharTable}{%
  \lst@ifbreaklines\lst@Def{`)}{\lst@breakProcessOther)}\fi}{}{}{}
\makeatother
\usepackage{hyperref}
\usepackage{todonotes}
\usepackage{makeidx}
\usepackage[inline]{enumitem}
\usepackage[francais]{babel}
\usepackage{caption, tabu}
\usepackage[headsepline=1pt,plainheadsepline,
  footsepline=1pt,plainfootsepline]{scrlayer-scrpage}  % header and footer for KOMA-Script

% ---- User defined commands
\newcommand{\N}{\mathbb{N}}
\newcommand{\Q}{\mathbb{Q}} 
\newcommand{\R}{\mathbb{R}}
\newcommand{\Z}{\mathbb{Z}}
\newcommand{\F}{\mathbb{F}}
\newcommand{\Fa}{\F(A)} 
\newcommand{\C}{\mathbb{C}}
\newcommand{\K}{\mathbb{K}}
\renewcommand{\a}{\mathfrak{a}} % notation idéal
\renewcommand{\epsilon}{\varepsilon}
\renewcommand{\phi}{\varphi}
\renewcommand{\emph}{\textbf}
\newcommand{\im}{\mathrm{Im}}
\newcommand{\pgdc}{\mathrm{pgdc}}
\newcommand{\ext}[2]{\overset{#1}{\underset{#2}{\text{\tiny{$|$}}}}}

% ---- Synchronization with Skim
\synctex=1


%%%%%%%%	Définitions des environnements de théorèmes	%%%%%%%%
%----- ENVIRONNEMENT POUR LES DÉFINITIONS ----%
\declaretheoremstyle[
  spaceabove=0pt, spacebelow=0pt, headfont=\normalfont\bfseries\scshape,
    notefont=\mdseries, notebraces={(}{)}, headpunct={. }, headindent={},
    postheadspace={ }, postheadspace=4pt, bodyfont=\normalfont, %qed=$$,
    mdframed={
      leftmargin=-5,
      rightmargin=-5,
      middlelinewidth=1pt,
      roundcorner=5pt,
      middlelinecolor=DarkSlateGrey,
      innerlinecolor=DarkSlateGrey,
      outerlinecolor=DarkSlateGrey,
      % apptotikzsetting={\tikzset{mdfbackground/.append style ={
      %       shade, left color=DarkSlateGrey!20, right color = DarkSlateGrey!20}}}
   }
]{defstyle}

\declaretheorem[style=defstyle, numberwithin=chapter, title=Définition]{defi}
%________________________________________________________


%----- ENVIRONNEMENT POUR LES THEOREMES ----%
\declaretheoremstyle[
  spaceabove=0pt, spacebelow=0pt, headfont=\normalfont\bfseries\scshape,
    notefont=\mdseries, notebraces={(}{)}, headpunct={. }, headindent={},
    postheadspace={ }, postheadspace=4pt, bodyfont=\normalfont\itshape, %qed=$$,
    mdframed={
      leftmargin=-5,
      rightmargin=-5,
      middlelinewidth=1pt,
      roundcorner=5pt,
      middlelinecolor=DarkOrange,
      innerlinecolor=DarkOrange,
      outerlinecolor=DarkOrange,
      % apptotikzsetting={\tikzset{mdfbackground/.append style ={
      %       shade, left color=DarkSlateGrey!20, right color = DarkSlateGrey!20}}}
   }
]{thmstyle}
\declaretheorem[style=thmstyle, sibling=defi, title=Théorème]{theo}
\declaretheorem[style=thmstyle, sibling=defi, title=Corollaire]{cor}
\declaretheorem[style=thmstyle, sibling=defi, title=Proposition]{prop}
\declaretheorem[style=thmstyle, sibling=defi, title=Propriétés]{propri}
\declaretheorem[style=thmstyle, sibling=defi, title=Observation]{obs}
\declaretheorem[style=thmstyle, sibling=defi, title=Observations]{obss}
\declaretheorem[style=thmstyle, sibling=defi, title=Lemme]{lem}
\declaretheorem[style=thmstyle, sibling=defi, title=Conséquence]{conseq}
%_________________________________________________________

%----- ENVIRONNEMENT POUR LES PREUVES ----%
\declaretheoremstyle[
  spaceabove=0pt, spacebelow=0pt, headfont=\normalfont\bfseries\scshape,
    notefont=\mdseries, notebraces={(}{)}, headpunct={. }, headindent={},
    postheadspace={ }, postheadspace=4pt, bodyfont=\normalfont, 
    mdframed={
      leftmargin=15,
      rightmargin=15,
      hidealllines=true,
      font=\small
   }
]{preuvestyle}

\declaretheorem[style=preuvestyle, numbered = no, title=Preuve, qed=\textcolor{DarkSlateGrey!80}{\qedsymbol}]{preuve}
\declaretheorem[style=preuvestyle, title=Exercice, numberwithin=chapter, qed=\textcolor{DarkSlateGrey!80}{$\spadesuit$}]{exercice}
\declaretheorem[style=preuvestyle, sibling=defi, title=Remarque, qed =
\textcolor{DarkSlateGrey!80}{$\clubsuit$}]{rem}
\declaretheorem[style=preuvestyle, sibling=defi, title=Remarques, qed = \textcolor{DarkSlateGrey!80}{$\clubsuit$}]{rems}
%________________________________________________________
%----- ENVIRONNEMENT POUR LES EXEMPLES ----%
\declaretheoremstyle[
  spaceabove=0pt, spacebelow=0pt, headfont=\normalfont\bfseries\scshape,
    notefont=\mdseries, notebraces={(}{)}, headpunct={. }, headindent={},
    postheadspace={ }, postheadspace=4pt, bodyfont=\normalfont, qed=\textcolor{DarkOliveGreen!80}{$\bigstar$},
    mdframed={
      leftmargin=15,
      rightmargin=15,
      font=\small,
      outerlinewidth=1pt,
      innerlinewidth=1pt,
      middlelinewidth=1pt,
      hidealllines=true,
      topline=true,
      bottomline=true,
      innerlinecolor=DarkOliveGreen!80,
      outerlinecolor=DarkOliveGreen!80,
      middlelinecolor=DarkOliveGreen!80
      % , leftline=true,
      %innerlinecolor=DarkSlateGrey!80,
      %outerlinecolor=DarkSlateGrey!80,
      %middlelinecolor=White,
   }
]{exstyle}

\declaretheorem[style=exstyle, numberlike=defi, title=Exemple]{ex}
\declaretheorem[style=exstyle, numberlike=defi, title=Exemples]{exs}
%________________________________________________________

% ---- Headers and footers
\clearpairofpagestyles                 % deletes header/footer
\pagestyle{scrheadings}           % use following definitions for header/footer
% definitions/configuration for the header
\lohead[Université de \textsc{Neuchâtel}]{Université de \textsc{Neuchâtel}}
\lehead[Université de \textsc{Neuchâtel}]{Université de \textsc{Neuchâtel}}        % equal page, right position (inner) 
\rohead[\leftmark]{\leftmark}        % odd   page, left  position (inner) 
\rehead[\rightmark]{\rightmark} % equal page, left (outer) position
% definitions/configuration for the footer
\lefoot[Théorie de Galois]{Théorie de Galois}
\lofoot[Théorie de Galois]{Théorie de Galois}
\refoot[page \pagemark]{page \pagemark}
\rofoot[page \pagemark]{page \pagemark}
\renewcommand{\chaptermark}[1]{%
  \markboth{#1}{}}


\addtokomafont{disposition}{\normalfont\bfseries}

\title{\normalfont{\bfseries{Théorie de Galois \\ {\Large printemps 2017} \\{\large Université de Neuchâtel}}}}
\author{Enseigné par Ana \textsc{Khukhro}\\Notes prises par Laurent \textsc{Hayez}}
\date{Date de création: 23 février 2017\\ Dernière modification: \today}

\makeindex

\begin{document}


\renewcommand{\labelitemi}{\textbullet}

\tikzset{math3d/.style=
{x= {(-0.353cm,-0.353cm)}, y={(1cm,0cm)}, z={(0cm,1cm)}}}


\maketitle


%Table of contents
\tableofcontents

\setcounter{chapter}{-1}

% Chapter 0: Introduction et histoire
% ********************************************************************************************************** %
% *                                            THÉORIE DE GALOIS                                           * %
% *                                            -----------------                                           * %
% *                                                                                                        * %
% * Chapitre 0: Introduction                                                                               * %
% *                                                                                                        * %
% * Auteur: Laurent Hayez                                                                                  * %
% * Cours enseigné au semestre de printemps 2017 à l'université de Neuchâtel par Ana Khukhro               * %
% ********************************************************************************************************** %

\chapter{Introduction et histoire}
Babylone vers 1600 av. J.-C., solution de l'équation du second degré. 
\[ ax^2 + bx + c = 0 \iff x = \frac{-b \pm \sqrt{b^2 - 4ac}}{2a}. \]
En $\sim\!$1540 ap. J.-C., Caradano et Ferrari donnent une solution pour les polynômes de degré jusqu'à $4$. 
La question restait ouverte pour les polynômes de degrés plus grands ou égal à $5$. \og Résolubles par
radicaux\fg{}? Évariste \textsc{Galois} (1811-1832) donne la solution. 

Si $p(x)$ est un polynôme et $E$ une extension d'un corps $K$ $\ext{E}{K}$, son idée est de construire un
groupe à partir de $E$.

Il y a également des constructions à la règle et au compas (Grèce: Euclide, etc. en 300/400
av. J.-C. environ), par exemple donner l'ensemble des points équidistants à un point $A$ et un point $B$. Il y
a des questions que les grecs n'ont pas réussi à résoudre, par exemple
\begin{itemize}
\item la trisection de l'angle (partager un angle en 3);
\item la quadrature du cercle;
\item la duplication du cube (cube donné, trouver un cube plus grand ayant le double du volume).
\end{itemize}





  




%%% Local Variables:
%%% mode: latex
%%% TeX-master: "../GAL_cours.tex" 
%%% End:

% Chapter 1: Présentations des groupes
% ********************************************************************************************************** %
% *                                            THÉORIE DE GALOIS                                           * %
% *                                            -----------------                                           * %
% *                                                                                                        * %
% * Chapitre 1: Rappels et notions basiques                                                                * %
% *                                                                                                        * %
% * Auteur: Laurent Hayez                                                                                  * %
% * Cours enseigné au semestre de printemps 2017 à l'université de Neuchâtel par Ana Khukhro               * %
% ********************************************************************************************************** %

\chapter{Rappels et notions basiques}


\begin{defi}
  Un \emph{anneau} est un ensemble $A$ muni de deux opérations (lois) de composition appelées respectivement
  addition et multiplication satisfaisant:
  \begin{itemize}
  \item pour l'addition, $A$ est un groupe commutatif;
  \item la multiplication est associative et possède un élément neutre (\emph{unité});
  \item la multiplication est distributive par rapport à l'addition.
  \end{itemize}

\end{defi}

Dans le reste du cours, tout anneau sera commutatif!

\begin{itemize}
\item $A^\ast$ est le groupe multiplicatif de $A$, c'est-à-dire l'ensemble des éléments inversibles par
  rapport à la multiplication.

\item Un \emph{sous-anneau} $B$ de $A$ est une partie de $A$ qui est un sous-groupe additif, qui est stable
  par multiplication, et contient l'élément neutre. Par exemple, $\Z$ et $\Q$ sont des anneaux, et comme $\Z
  \subset \Q$, $\Z$ est un sous-anneau de $\Q$.

\item Un \emph{idéal} $\a$ de $A$ est un sous-groupe du groupe additif de $A$ tel que pour tout $x \in A$ et $a
  \in \a$, alors $xa \in \a$. Par exemple $n\Z \subset \Z$ est un idéal de $\Z$ pour tout $n \in \N$.

\item L'\emph{idéal principal} engendré par $a \in A$ est $Aa = \{xa \mid x \in A\} =: (a)$.

\item Un \emph{anneau principal} est un anneau \emph{intègre} (si $ab = 0$, alors $a = 0$ ou $b = 0$, i.e.,
  pas de diviseur de $0$) où tout idéal est principal.

\item Un idéal $p \neq A$ de $A$ est dit \emph{premier} si les conditions équivalentes suivantes sont satisfaites:
  \begin{itemize}
  \item l'anneau $A/p$ est intègre;
  \item pour tous $x, y \in A$ et $xy \in p$, alors $x \in p$ ou $y \in p$.
  \item $p$ est le noyau d'un homomorphisme de $A$ dans un corps.
  \end{itemize}

\item $m$ est un \emph{idéal maximal} si $m \neq A$ et $m \subset \a$ un autre idéal, alors $\a = m$.

\item Si $m$ est maximal, alors $m$ est premier. La réciproque est vraie dans un anneau principal, un idéal
  non-nul premier est maximal.

\item Chaque $\a \neq A$ est contenu dans un idéal maximal.

\item Un anneau $\K \neq 0$ est un \emph{corps} si tout élément non-nul de $\K$ est inversible.

\item Soit $A$ un anneau, $\a$ un idéal de $A$. $\a$ est premier ssi $A/\a$ est intègre et $\a$ est maximal ssi
  $A/\a$ est un corps.

\item Soit $\K$ un corps. $\K[X]$ est l'anneau des polynômes à coefficients dans $\K$. $\K[X]$ est un anneau
  principal. 

\item Les idéaux premiers de $\K[X]$ sont
  \begin{itemize}
  \item $(0)$ (car $\K[X]$ est intègre);
  \item $(f)$ pour $f \in \K[X]$ est un polynôme irréductible (un élément $a \in A$, $a \neq 0$, anneau intègre, est
    \emph{irréductible} si $a$ n'est pas inversible et si $a = bc$, alors $b$ ou $c$ est inversible).
  \end{itemize}

\item Un polynôme est irréductible s'il est non-constant et il n'est pas un produit de deux polynômes
  non-constants de degrés inférieurs.

\item Tout idéal premier non-nul de $\K[X]$ est maximal.

\item Sont équivalentes pour $f \neq 0 \in \K[X]$:
  \begin{itemize}
  \item $f$ est irréductible;
  \item $(f)$ est premier;
  \item $\K[X]/(f)$ est intègre;
  \item $(f)$ est maximal;
  \item $\K[X]/(f)$ est un corps.
  \end{itemize}


\end{itemize}

\begin{exs}
  \begin{itemize}
  \item $\K = \R$, $f(X) = X^2 + 1$. Alors $\R[X]/(f) \simeq \C$.
  \item $\K = \F_2$, $f(X)= X^2 + X + 1$, alors $\F_2[X]/(f) \simeq \F_4$.
  \end{itemize}
\end{exs}

\begin{itemize}
\item Un anneau $A$ est dit \emph{factoriel} si $A$ est intègre et tout élément $a \neq 0 \in A$ s'écrit comme
  produit 
  \[ a = u \prod_{i \in I} p_i \]
  où $u \in A^\ast$ et $\{p_i \mid i \in I\}$ est un ensemble fini d'éléments irréductibles (unique à
  multiplication près).

\item Tout anneau principal est factoriel ($\Z$, $\K[X]$, ...).

\item Si $A$ est factoriel, alors $A[X]$ l'est aussi.

\item Les éléments irréductibles de $A[X]$ sont les éléments irréductibles de $A$ et les polynômes
  non-constant avec $\pgdc$ des coefficients égal à 1, qui restent irréductibles dans $\K[X]$, où $\K$ est le
  corps de fractions de $A$.

\item Dans un anneau factoriel, un élément irréductible $p$ engendre un idéal premier $\mathfrak{p} = (p)$. 

\end{itemize}




% Critères d'irréductibilité

\section{Critères d'irréductibilité}

Soit $A$ un anneau factoriel et soit $\K$ son corps de fractions. 

\paragraph{Critère d'Eisenstein:}

soit $f(X) = a_nX^n + \cdots + a_0$ un polynôme de degré $n \geq 1$ dans $A[X]$. Soit $p$ un
élément irréductible de $A$. Si $a_n \not \equiv 0 \pmod p$, $a_i \equiv 0 \pmod p$ pour tout $i < n$ et $a_0
\not \equiv 0 \pmod{p^2}$, alors $f(X)$ est irréductible dans $\K[X]$.

\begin{ex}
  Soit $f(X) = \frac{2}{9} X^5 + \frac{5}{3} X^4 + X^3 + \frac{1}{3} \in \Q[X]$. En multipliant par $9$, on
  obtient un polynôme dans $\Z[X]$. Ainsi $f(X)$ est irréductible si et seulement si $9f(X) = 2X^5 + 15X^4 +
  9X^3 + 3$ est irréductible. Par le critère d'Eisenstein pour $p = 3$, $f(X)$ est irréductible dans $\Q[X]$.
\end{ex}

\paragraph{Réduction:}

soit $f(X) = a_n X^n + \cdots + a_0$ monique ($a_n = 1$) et soit $p \in A$ irréductible. Soit $\overline{f}$
l'image de $f$ dans $A/(p)[X]$. Si $\overline{f}$ est irréductible dans $A/(p)[X]$, il l'est aussi dans $A[X]$
(et aussi $\K[X]$).

\begin{ex}
  Soit $f(X) = X^3 + 2X^2 + X + 5 \in \Q[X]$. On prend $p = 2$. $\overline{f}(X) = X^3 + X + 1 \in
  \Z/2\Z[X]$. On remarque que s'il existait une factorisation non triviale de $\overline{f}$, alors l'un des
  polynôme serait de la forme $(X - \xi)$, i.e., il admettrait une racine. Comme on vérifie facilement qu'il
  n'en possède pas, $\overline{f}$ est irréductible, et donc $f$ aussi.
\end{ex}

\paragraph{Dérivation et racines multiples:}

Soit $A$ un anneau. On définit la dérivation par $D: A[X] \to A[X]$, $a_nX^n + \cdots + a_0 \mapsto
na_nX^{n-1} + \cdots + a_1$.
\begin{itemize}
\item $D$ est $A$-linéaire.
\item $D(fg) = D(f)g + fD(g)$.
\item $D((X-a)^m) = m(X-a)^{m-1}$.
\end{itemize}

\begin{defi}
  Soit $\K$ un corps et soit $f \in \K[X]$. Soit $a \in \K$ une racine de $f$. On peut écrire $f(X) =
  (X-a)^{m}g(X)$ où $g(X)$ est premier avec $(X-a)$, et $m$ est appelée la \emph{multiplicité} de $a$ et on
  dit que $a$ une \emph{racine multiple} si $m > 1$.
\end{defi}

\begin{prop}
  Un élément $a \in \K$ est une racine multiple de $f$ ssi $a$ est une racine de $f$ et $D(f)(a) = 0$.
\end{prop}

\begin{preuve}
  Exercice.
\end{preuve}



% Caractéristique d'un corps

\section{Caractéristique d'un corps}
\label{sec:caract-dun-corps}

Soit $\K$ un corps. On considère l'homomorphisme d'anneau 
\[
  \begin{array}{cccl}
    \eta: & \Z & \to & \K \\
    & n & \mapsto & \mathrm{sgn}(n) \cdot (\underbrace{1 + 1 + \cdots + 1}_{n \text{ fois}})
  \end{array}
\]

$\ker(\eta)$ est un idéal premier de $\Z$, car $\Z/\ker(\eta) \simeq \mathrm{Im}(\eta) \subset \K$ est un
anneau intègre. Il y a deux cas:
\begin{itemize}
\item $\ker(\eta) = \{0\}$ et donc $\eta$ est injective, $\Z$ est un sous-anneau de $\K$, et $\K$ contient le
  corps de fractions de $\Z$. Dans ce cas, on dit que $\K$ est de \emph{caractéristique} $0$.

\item $\ker(\eta) = p\Z$, $p$ premier. $p$ est la \emph{caractéristique} de $\K$. Dans ce cas, $\F_p$ est un
  sous-corps de $\K$ et $\underbrace{1 + 1 + \cdots + 1}_{p \text{ fois}} = 0$ dans $\K$.
\end{itemize}


















%%% Local Variables:
%%% mode: latex
%%% TeX-master: "../GAL_cours.tex" 
%%% End:

% Chapter 2: 
%% ********************************************************************************************************** %
% *                                            THÉORIE DE GALOIS                                           * %
% *                                            -----------------                                           * %
% *                                                                                                        * %
% * Chapitre 2:                                                                                * %
% *                                                                                                        * %
% * Auteur: Laurent Hayez                                                                                  * %
% * Cours enseigné au semestre de printemps 2017 à l'université de Neuchâtel par Ana Khukhro               * %
% ********************************************************************************************************** %

\chapter{}






%%% Local Variables:
%%% mode: latex
%%% TeX-master: "../GAL_cours.tex" 
%%% End:




\printindex
	
\end{document}



%%% Local Variables:
%%% mode: latex
%%% TeX-master: t 
%%% End: