% ********************************************************************************************************** %
% *                                            THÉORIE DE GALOIS                                           * %
% *                                            -----------------                                           * %
% *                                                                                                        * %
% * Chapitre 1: Rappels et notions basiques                                                                * %
% *                                                                                                        * %
% * Auteur: Laurent Hayez                                                                                  * %
% * Cours enseigné au semestre de printemps 2017 à l'université de Neuchâtel par Ana Khukhro               * %
% ********************************************************************************************************** %

\chapter{Rappels et notions basiques}


\begin{defi}
  Un \emph{anneau} est un ensemble $A$ muni de deux opérations (lois) de composition appelées respectivement
  addition et multiplication satisfaisant:
  \begin{itemize}
  \item pour l'addition, $A$ est un groupe commutatif;
  \item la multiplication est associative et possède un élément neutre (\emph{unité});
  \item la multiplication est distributive par rapport à l'addition.
  \end{itemize}

\end{defi}

Dans le reste du cours, tout anneau sera commutatif!

\begin{itemize}
\item $A^\ast$ est le groupe multiplicatif de $A$, c'est-à-dire l'ensemble des éléments inversibles par
  rapport à la multiplication.

\item Un \emph{sous-anneau} $B$ de $A$ est une partie de $A$ qui est un sous-groupe additif, qui est stable
  par multiplication, et contient l'élément neutre. Par exemple, $\Z$ et $\Q$ sont des anneaux, et comme $\Z
  \subset \Q$, $\Z$ est un sous-anneau de $\Q$.

\item Un \emph{idéal} $I$ de $A$ est un sous-groupe du groupe additif de $A$ tel que pour tout $x \in A$ et $a
  \in I$, alors $xa \in I$. Par exemple $n\Z \subset \Z$ sont des idéaux de $\Z$ pour tout $n \in \N$.

\item L'\emph{idéal principal} engendré par $a \in A$ est $Aa = \{xa \mid x \in A\} =: (a)$.

\item Un \emph{anneau principal} est un anneau \emph{intègre} (si $ab = 0$, alors $a = 0$ ou $b = 0$, i.e.,
  pas de diviseur de $0$) où tout idéal est principal.

\item Un idéal $p \neq A$ de $A$ est dit \emph{premier} si les conditions équivalentes suivantes sont satisfaites:
  \begin{itemize}
  \item l'anneau $A/p$ est intègre;
  \item pour tous $x, y \in A$ et $xy \in p$, alors $x \in p$ ou $y \in p$.
  \item $p$ est le noyau d'un homomorphisme de $A$ dans un corps.
  \end{itemize}

\item $m$ est un \emph{idéal maximal} si $m \neq A$ et $m \subset I$ un autre idéal, alors $I = m$.

\item Si $m$ est maximal, alors $m$ est premier. La réciproque est vraie dans un anneau principal, un idéal
  non-nul premier est maximal.

\item Chaque $I \neq A$ est contenu dans un idéal maximal.

\item Un anneau $K \neq 0$ est un \emph{corps} si tout élément non-nul de $K$ est inversible.

\item Soit $A$ un anneau, $I$ un idéal de $A$. $I$ est premier ssi $A/I$ est intègre et $I$ est maximal ssi
  $A/I$ est un corps.

\item Soit $K$ un corps. $K[X]$ est l'anneau des polynômes à coefficients dans $K$. $K[X]$ est un anneau
  principal. 

\item Les idéaux premiers de $K[X]$ sont
  \begin{itemize}
  \item $(0)$ (car $K[X]$ est intègre);
  \item $(f)$ pour $f \in K[X]$ est un polynôme irréductible (un élément $a \in A$, $a \neq 0$, anneau intègre, est
    \emph{irréductible} si $a$ n'est pas inversible et si $a = bc$, alors $b$ ou $c$ est inversible).
  \end{itemize}

\item Un polynôme est irréductible s'il est non-constant et il n'est pas un produit de deux polynômes
  non-constants de degrés inférieurs.

\item Tout idéal premier non-nul de $K[X]$ est maximal.

\item Sont équivalentes pour $f \neq 0 \in K[X]$:
  \begin{itemize}
  \item $f$ est irréductible;
  \item $(f)$ est premier;
  \item $K[X]/(f)$ est intègre;
  \item $(f)$ est maximal;
  \item $K[X]/(f)$ est un corps.
  \end{itemize}


\end{itemize}

\begin{exs}
  \begin{itemize}
  \item $K = \R$, $f(X) = X^2 + 1$. Alors $\R[X]/(f) \simeq \C$.
  \item $K = \F_2$, $f(X)= X^2 + X + 1$, alors $\F_2[X]/(f) \simeq \F_4$.
  \end{itemize}
\end{exs}

\begin{itemize}
\item Un anneau $A$ est dit \emph{factoriel} si $A$ est intègre et tout élément $a \neq 0 \in A$ s'écrit comme
  produit 
  \[ a = u \prod_{i \in I} p_i \]
  où $u \in A^\ast$ et $\{p_i \mid i \in I\}$ est un ensemble fini d'éléments irréductibles (unique à
  multiplication près).

\item Tout anneau principal est factoriel ($\Z$, $K[X]$, ...).

\item Si $A$ est factoriel, alors $A[X]$ l'est aussi.

\item Les éléments irréductibles de $A[X]$ sont les éléments irréductibles de $A$ et les polynômes
  non-constant avec $\pgdc$ des coefficients égal à 1, qui restent irréductibles dans $K[X]$, où $K$ est le
  corps de fractions de $A$.

\item Dans un anneau factoriel, un élément irréductible $p$ engendre un idéal premier. 

\end{itemize}




% Critères d'irréductibilité

\section{Critères d'irréductibilité}

Soit $A$ un anneau factoriel et soit $K$ son corps de fractions. 

\paragraph{Critère d'Eisenstein:}

soit $f(X) = a_nX^n + \cdots + a_0$ un polynôme de degré $n \geq 1$ dans $A[X]$. Soit $p$ un
élément irréductible de $A$. Si $a_n \not \equiv 0 \pmod p$, $a_i \equiv 0 \pmod p$ pour tout $i < n$ et $a_0
\not \equiv 0 \pmod{p^2}$, alors $f(X)$ est irréductible dans $K[X]$.

\begin{ex}
  Soit $f(X) = \frac{2}{9} X^5 + \frac{5}{3} X^4 + X^3 + \frac{1}{3} \in \Q[X]$. En multipliant par $9$, on
  obtient un polynôme dans $\Z[X]$. Ainsi $f(X)$ est irréductible si et seulement si $9f(X) = 2X^5 + 15X^4 +
  9X^3 + 3$ est irréductible. Par le critère d'Eisenstein pour $p = 3$, $f(X)$ est irréductible dans $\Q[X]$.
\end{ex}

\paragraph{Réduction:}

soit $f(X) = a_n X^n + \cdots + a_0$ monique ($a_n = 1$) et soit $p \in A$ irréductible. Soit $\overline{f}$
l'image de $f$ dans $A/(p)[X]$. Si $\overline{f}$ est irréductible dans $A/(p)[X]$, il l'est aussi dans $A[X]$
(et aussi $K[X]$).

\begin{ex}
  Soit $f(X) = X^3 + 2X^2 + X + 5 \in \Q[X]$. On prend $p = 2$. $\overline{f}(X) = X^3 + X + 1 \in
  \Z/2\Z[X]$. On remarque que s'il existait une factorisation non triviale de $\overline{f}$, alors l'un des
  polynôme serait de la forme $(X - \xi)$, i.e., il admettrait une racine. Comme on vérifie facilement qu'il
  n'en possède pas, $\overline{f}$ est irréductible, et donc $f$ aussi.
\end{ex}

\paragraph{Dérivation et racines multiples:}

Soit $A$ un anneau. On définit la dérivation par $D: A[X] \to A[X]$, $a_nX^n + \cdots + a_0 \mapsto
na_nX^{n-1} + \cdots + a_1$.
\begin{itemize}
\item $D$ est $A$-linéaire.
\item $D(fg) = D(f)g + fD(g)$.
\item $D((x-a)^m) = m(x-a)^{m-1}$.
\end{itemize}

\begin{defi}
  Soit $K$ un corps et soit $f \in K[X]$. Soit $a \in K$ une racine de $f$. On peut écrire $f(X) =
  (X-a)^{m}g(X)$ où $g(X)$ est premier avec $(X-a)$, et $m$ est appelée la \emph{multiplicité} de $a$ et on
  dit que $a$ une \emph{racine multiple} si $m > 1$.
\end{defi}

\begin{prop}
  Un élément $a \in K$ est une racine multiple de $f$ ssi $a$ est une racine de $f$ et $D(f)(a) = 0$.
\end{prop}

\begin{preuve}
  Exercice.
\end{preuve}



% Caractéristique d'un corps

\section{Caractéristique d'un corps}
\label{sec:caract-dun-corps}

Soit $K$ un corps. On considère l'homomorphisme d'anneau 
\[
  \begin{array}{cccc}
    \eta & \Z & \to & K \\
    & n & \mapsto & \mathrm{sgn}(n) \cdot (\underbrace{1 + 1 + \cdots + 1}_{n \text{ fois}})
  \end{array}
\]

$\ker(\eta)$ est un idéal premier de $\Z$, car $\Z/\ker(\eta) \simeq \mathrm{Im}(\eta) \subset K$ est un
anneau intègre. Il y a deux cas:
\begin{itemize}
\item $\ker(\eta) = \{0\}$ et donc $\eta$ est injective, $\Z$ est un sous-anneau de $K$, et $K$ contient le
  corps de fractions de $\Z$. Dans ce cas, on dit que $K$ est de \emph{caractéristique} $0$.

\item $\ker(\eta) = p\Z$, $p$ premier. $p$ est la \emph{caractéristique} de $K$. Dans ce cas, $\F_p$ est un
  sous-corps de $K$ et $\underbrace{1 + 1 + \cdots + 1}_{p \text{ fois}} = 0$ dans $K$.
\end{itemize}


















%%% Local Variables:
%%% mode: latex
%%% TeX-master: "../GAL_cours.tex" 
%%% End: