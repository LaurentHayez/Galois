% ********************************************************************************************************** %
% *                                            THÉORIE DE GALOIS                                           * %
% *                                            -----------------                                           * %
% *                                                                                                        * %
% * Chapitre 0: Introduction                                                                               * %
% *                                                                                                        * %
% * Auteur: Laurent Hayez                                                                                  * %
% * Cours enseigné au semestre de printemps 2017 à l'université de Neuchâtel par Ana Khukhro               * %
% ********************************************************************************************************** %

\chapter{Introduction et histoire}
Babylone vers 1600 av. J.-C., solution de l'équation du second degré. 
\[ ax^2 + bx + c = 0 \iff x = \frac{-b \pm \sqrt{b^2 - 4ac}}{2a}. \]
En $\sim\!$1540 ap. J.-C., Caradano et Ferrari donnent une solution pour les polynômes de degré jusqu'à $4$. 
La question restait ouverte pour les polynômes de degrés plus grands ou égal à $5$. \og Résolubles par
radicaux\fg{}? Évariste \textsc{Galois} (1811-1832) donne la solution. 

Si $p(x)$ est un polynôme et $E$ une extension d'un corps $K$ $\ext{E}{K}$, son idée est de construire un
groupe à partir de $E$.

Il y a également des constructions à la règle et au compas (Grèce: Euclide, etc. en 300/400
av. J.-C. environ), par exemple donner l'ensemble des points équidistants à un point $A$ et un point $B$. Il y
a des questions que les grecs n'ont pas réussi à résoudre, par exemple
\begin{itemize}
\item la trisection de l'angle (partager un angle en 3);
\item la quadrature du cercle;
\item la duplication du cube (cube donné, trouver un cube plus grand ayant le double du volume).
\end{itemize}





  




%%% Local Variables:
%%% mode: latex
%%% TeX-master: "../GAL_cours.tex" 
%%% End: